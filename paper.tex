%% Latest change: Thu Mar 11 18:06:33 CET 2010
\documentclass[a4paper]{scrartcl}

\usepackage{ucs}
\usepackage[utf8x]{inputenc}
\usepackage{amsmath}

\usepackage{math} % Prof. Paulus' convenient math symbols

\begin{document}

\title{Math To Remember}
\author{Jonas Münch}
\date{\today}

\maketitle

\tableofcontents

%\begin{abstract}
%This is my abstract.
%\end{abstract}

\section{Gauß-Funktion}
\[
  G_{\sigma, \mu} (x) = \frac{1}{\sqrt{2 \pi \sigma^2}}e^{-\frac{(x - \mu)^2}{2 \sigma^2}}
\]

\paragraph{Multidimensional}
Mit der Kovarianzmatrix $\mat{\Sigma}$ und dem Erwartungswertvektor $\vec{\mu}$ im $n$-dimensionalen:
\[
  G_{\mat{\Sigma}, \vec{\mu}} (\vec{x}) = 
  \frac{1}{\sqrt{(2 \pi)^n \determinant{\mat{\Sigma}}}}\funv{exp}\left( -\half \transpose{(\vec{x}-\vec{\mu})} \mat{\Sigma}^{-1} (\vec{x}-\vec{\mu})\right)
\]
$\transpose{(\vec{x}-\vec{\mu})} \mat{\Sigma}^{-1} (\vec{x}-\vec{\mu})$ ist hierbei die \emph{Mahalanobis-Distanz} und gibt die quadratische statistische Distanz von \vec{x} zu \vec{\mu} an.

\section{Bayes-Formel}

\[
  P(B|A) = \frac{P(B) \cdot P(A|B)}{P(A)}
\]

\section{Ableitungsregeln}

\paragraph{Produktregel}
\begin{align*}
    f(x) &= g(x) \cdot h(x) \\
    f'(x) &= g'(x) \cdot h(x) + g(x) \cdot h'(x)
\end{align*}

\paragraph{Quotientenregel}
\begin{align*}
    f(x) &= \frac{g(x)}{h(x)} \\
    f'(x) &= \frac{g'(x) \cdot h(x) - g(x) \cdot h'(x)}{h(x)^2}
\end{align*}

\paragraph{Kettenregel}
\begin{align*}
    f(x) &= g((h(x)) \\
    f'(x) &= g'(h(x)) \cdot h'(x)
\end{align*}


\section{Integrationsregeln}

\paragraph{Partielle Integration / Produktintegration}
\begin{align*}
  \int u(x) \cdot v(x) dx &= u(x) \cdot v(x) - \int u'(x) \cdot v(x) dx
\end{align*}

\section{Jacobi-Matrix}

Für eine Funktion $ f: \defunran{\real^n}{\real^m}$ gibt die \emph{Jacobi-Matrix}
\[
  \mat{J}_f(x) = \begin{Matrix}(3, 3)
    \frac{\partial f_1}{\partial x_1}
    &
    \ldots
    & 
    \frac{\partial f_1}{\partial x_n}
  \\ \vdots & \ddots & \vdots \\
    \frac{\partial f_m}{\partial x_1}
    &
    \ldots
    &
    \frac{\partial f_m}{\partial x_n}
  \end{Matrix}
\]
die ersten partiellen Ableitungen an.

\section{Gradient, Divergenz, Laplace-Operator}

\paragraph{Gradient}
\begin{align*}
&  f: \defunvar{\real^2}{\real}
  \\
&  \nabla f(x, y) = \colvec(\frac{\partial f(x, y)}{\partial x}, \frac{\partial f(x, y)}{\partial y})
\end{align*}

\paragraph{Divergenz}
\begin{align*}
&  f: \defunvar{\real^2}{\real^2} \textrm{(Vektorfeld)}
  \\
&  \nabla f(x, y) = \frac{\partial f_x(x, y)}{\partial x} + \frac{\partial_y f(x, y)}{\partial y}
\end{align*}

\paragraph{Laplace Operator}
\begin{align*}
&  f: \defunvar{\real^2}{\real}
  \\
&  \nabla^2 = \bigtriangleup f(x, y) = \colvec(\frac{\partial^2 f(x, y)}{\partial x^2}, \frac{\partial^2 f(x, y)}{\partial y^2})
\end{align*}

\section{Strukturtensor}

Der \emph{Strukturtensor} bildet sich aus dem Gradienten $\nabla$
\[
  J_0(\nabla_\varphi) := \nabla \varphi \cdot \transpose{\nabla \varphi}
\]
Seine Eigenvektoren/-werte haben folgende Eigenschaften:
\begin{align*}
&  \vec{v}_1 \parallel \nabla \varphi &  \lambda_1 &= \| \nabla \varphi  \|^2 \\
&  \vec{v}_2 \perp \nabla \varphi &  \lambda_2 &= 0 \\
\end{align*}

\section{Hesse-Matrix}

Für eine Funktion $f:\defunran{\real^n}{\real}$ werden die zweiten partiellen ableitungen mit der \emph{Hesse-Matrix} angegeben. Hier das Beispiel für 2D:
\[
  \mat{H}_f(i\vec{x}) = \begin{Matrix}(2, 2) 
    \frac{\partial^2 f}{\partial x_1 \partial x_1} 
    & 
    \frac{\partial^2 f}{\partial x_1 \partial x_2} 
  \\
    \frac{\partial^2 f}{\partial x_1 \partial x_2} 
    &
    \frac{\partial^2 f}{\partial x_2 \partial x_2} 
  \end{Matrix}
\]


\section{Mapping}

Folgende Funktion bildet einen Wert $v$ von $[a_{in}, b_{in}]$ auf $[a_{out}, b_{out}]$ ab.

\[
  \funv{map}(v, a_{in}, b_{in}, a_{out}, b_{out})
  =
  \frac{ (v - a_{in}) }{ (b_{in} - a_{in}) } \cdot (b_{out} - a_{out}) + a_{out}
\]

\section{Rotation mit Quaternionen}

Wird ein Punkt $\vec{p}$ als Quaternion $\vec{p}^q$ ausgedrückt, so kann er mit dem Rotationsquaternion $r$ rotiert werden.

\begin{align*}
  r &= \left( \cos\left(\frac{\theta}{2}\right), \sin\left(\frac{\theta}{2}\right) \cdot \transpose{\vec{u}} \right) & \textrm{mit \transpose{\vec{u}}: rot.-Achse, $\theta$: rot.-Winkel}
\\
  \vec{p}^q &= \left( 0, \transpose{\vec{p}} \right) & 0: \textrm{arbitrary scalar factor}
\\
\vec{p}^q_{\textrm{rot}} &= r \cdot \vec{p}^q \cdot \bar{r} & \bar{r} = w-xi-yj-zk \textrm{(konjugiertes Quaternion)}
\end{align*}

\section{Rotationsmatrizen}
\paragraph{2D}

\[
  \mat{R} = \begin{Matrix}(2, 2) \cos \alpha & -\sin \alpha \\ \sin \alpha & \cos \alpha \end{Matrix}
\]

\paragraph{3D}
\begin{align*}
  \mat{R}_x =
  \begin{Matrix}(3, 3)
    1 & 0 & 0 \\
    0 & \cos \alpha & - \sin \alpha \\
    0 & \sin \alpha & \cos \alpha
  \end{Matrix}, 
&
  \mat{R}_y =
  \begin{Matrix}(3, 3)
    \cos \alpha & 0 & \sin \alpha \\
    0 & 1 & 0 \\
    - \sin \alpha & 0 & \cos \alpha
  \end{Matrix}, 
&
  \mat{R}_z =
  \begin{Matrix}(3, 3)
    \cos \alpha & - \sin \alpha  & 0 \\
    \sin \alpha & \cos \alpha & 0 \\
    0 & 0 & 1
  \end{Matrix}
\end{align*}

\section{Kreuzproduktmatrix}

Das Kreuzprodukt zweier Vektoren $\vec{v} \vecprod \vec{w}$ kann durch die \emph{Kreuzproduktmatrix} $[\vec{v}]_{\vecprod}$ dargestellt werden.
\begin{align*}
  \vec{v} \vecprod \vec{w} &= [\vec{v}]_{\vecprod} \vec{w}\\
  \colvec(v_1, v_2, v_3) \vecprod \colvec(w_1, w_2, w_3) &=
  \begin{Matrix}(3,3)
    0 & -v_3 & v_2 \\ v_3 & 0 & -v_1 \\ -v_2 & v_1 & 0
  \end{Matrix}
  \colvec(w_1, w_2, w_3) &=
  \colvec(-v_3w_2 + v_2w_3, v_3w_1 - v_1w_3, -v_2w_1 + v_1w_2)
\end{align*}
  
\section{Vektormultiplikation}
\begin{align*}
  \transpose{\vec{a}} \vec{a} &= a & \textnormal{(Skalarprodukt)}
\\
  &= a_1^2 + a_2^2 + a_3^2
\\
  \vec{a} \transpose{\vec{a}} &= \mat{M}
\\
  &= \colvec(a_1, a_2, a_3) \rowvec(a_1, a_2, a_3) = 
  \begin{Matrix}(3, 3)
    a_1^2 & a_1a_2 & a_1a_3 \\
    a_1a_2 & a_2^2 & a_2a_3 \\
    a_1a_3 & a_2a_3 & a_3^3
  \end{Matrix}
\end{align*}


\section{Kovarianzmatrix}
\begin{align*}
  \vec{\mu} &= \frac{1}{n} \sum_{i=0}^n \vec{x}_i   & \textnormal{Erwartungswert}
\\
  \mat{\Sigma}(\vec{x}) &= \frac{1}{n} \sum_{i=0}^n (\vec{x}_i-\vec{\mu})\transpose{(\vec{x}_i-\vec{\mu})}
  & \textnormal{Kovarianzmatrix}
\end{align*}

\paragraph{Eigenwerte}

Wird die Kovarianzmatrix in ihre Eigenwerte $\lambda_i$ und Eigenvektoren $\vec{e}_i$ zerlegt,
\[
  \mat{\Sigma} \vec{e}_i = \lambda_i \vec{e}_i
\]
so sind die
\begin{itemize}
  \item Eigenvektoren $\vec{e}_i$ die Achsen des aufgespannten Ellipsoids
  \item Eigenwerte $\lambda_i$ proportional zur Länge der Achsen
\end{itemize}

\section{Hauptkomponentenanalyse (PCA)}

\section{Ebene durch Punkte (Plane fitting)}

Um eine Ebene mit minimalem quadratischem Fehler durch eine Menge von Punkten zu legen, wird die Matrix $ \mat{A} = \transpose{(\vec{p}_1 \vec{p}_2 \ldots \vec{p}_n)}$ mit $n \geq 3$ zentrierten Punkten $\vec{p}_i := \vec{p}_i - \vec{\mu}$ aufgestellt. 
Die Singulärwertzerlegung $\mat{A} = \mat{U} \mat{\Sigma} \transpose{\mat{V}}$ wird angewandt, sodass die Ebene von den ersten beiden Spaltenvektoren der Matrix \mat{U} aufgespannt wird und der dritte Spaltenvektor die Normale darstellt.

Alternativ kann die Matrix \mat{A} mit den projektiven Punkten $\tilde{\vec{p}_i}$ aufgestellt werden. Nun bildet der Spaltenvektor aus \mat{V}, der zum kleinsten Singulärwert gehört, die Normale der Ebene.

\section{Homografie}
%TODO
Siehe BV3 Vorlesungsfolien Kapitel 10

\section{SVD (Singulärwertzerlegung)}

Jede Matrix $\mat{A} \in \real^{n \vecprod m} $ kann in
\[
  \mat{A} = \mat{U} \mat{\Sigma} \transpose{\mat{V}}
\]
zerlegt werden, wobei
\begin{itemize}
  \item $ \mat{U} \in \real^{n \vecprod n} $ und $ \mat{V} \in \real^{m \vecprod m} $ orthogonal
  \item $ \mat{\Sigma} \in \real^{n \vecprod m}  = 
    \begin{Matrix}(4, 4)
      \sigma_1 & 0 & 0 & 0 \\
      0 & \sigma_2 & 0 & 0 \\
      0 & 0 & \ddots & \vdots \\
      0 & 0 & \ldots & \sigma_{min(m,n)}
    \end{Matrix} $

  \item $ \sigma_i $ sind die Singulärwerte $ \sigma_1 \geq \sigma_2 \geq \ldots \geq \sigma_{min(m,n)} \geq 0 $
\end{itemize}

Die SVD kann für viele Probleme der linearen Algebra genutzt werden. Hier einige Eigenschaften:
\begin{itemize}
  \item Rang von \mat{A}: $\funv{rank}(\mat{A}) = \#\{\sigma_i > 0\}$
  \item Kern von \mat{A} (Nullraum): Aufgespannt von den Spaltenvektoren $\vec{v}_i$ von \mat{V} deren $\sigma_i = 0 $ 
  \item
  \item \vdots TODO
\end{itemize}

\section{Eigenwerte 2x2 Matrix}

Die Eigenwerte einer $2 \vecprod 2$-Matrix
$
  \mat{A} = \begin{Matrix}(2,2) a & b \\ c & d \end{Matrix}
$
können in einer geschlossenen Form berechnet werden:
\begin{align*}
  \mat{A} \vec{v} &= \lambda \vec{v} \\
  \begin{Matrix}(2,2) a & b \\ c & d \end{Matrix} \colvec(u, v) &= \lambda \colvec(u, v)
\\
  \begin{Matrix}(2,2) a & b \\ c & d \end{Matrix} \colvec(u, v) &= \begin{Matrix}(2,2) \lambda & 0 \\ 0 & \lambda \end{Matrix} \colvec(u, v)
\\
  \left( \begin{Matrix}(2,2) a & b \\ c & d \end{Matrix} - \begin{Matrix}(2,2) \lambda & 0 \\ 0 & \lambda \end{Matrix} \right) \colvec(u, v) &= 0
\\ \vdots \\
  \lambda ^2 - \funv{sp}(\mat{A}) \lambda + \determinant{\mat{A}} &= 0
\\ \\
  \lambda_{1/2} &= \half \left( \funv{sp}(\mat{A}) \pm \sqrt{\funv{sp}(\mat{A})^2 - 4\determinant{\mat{A}}} \right)
\end{align*}
Die Eigenvektoren lassen sich dann folgendermaßen bestimmen:
\[
  (\mat{A} - \lambda_{1/2} \mat{I}) \cdot \vec{x} = \vec{0}
\]

\section{Koordinatentransformation}

\section{Taylor-Approximation}

A $k$ times differentiable function can be approximated by polynomials.

\begin{align*}
  Tf(x, a) &= \sum_{n=0}^\infty \frac{f^{(n)}(a)}{n!}(x-a)^n
\\
  &= f(a) + f'(a)(x-a) + \frac{f''(a)}{2}(x-a)^2 + \frac{f'''(a)}{6}(x-a)^3 + \dots
\end{align*}

\section{Fourier-Transformation}

Jede periodische Funktion
\[
  f(x) = f(x + T), T = 2\pi
\]

kann in $\sin$ und $\cos$ Funktionen zerlegt werden.

\[
  f(x) = \frac{a_0}{2} + \sum_{k=1}^\infty a_k \cos(kx) + b_k \sin(kx)
\]

Die Koeffizienten können mit
\begin{align*}
  a_0 &= \frac{1}{\pi} \int_0^{2\pi} f(x) dx \\
  a_k &= \frac{1}{\pi} \int_0^{2\pi} f(x) \cos(x) dx \\
  b_k &= \frac{1}{\pi} \int_0^{2\pi} f(x) \sin(x)dx
\end{align*}
bestimmt werden.

In Complexen Zahlen ausgedrückt:
\[
  F(u) = \int_{-\infty}^\infty f(x) e^{-2 \pi i u x} \funv{d}ux
\]

Inverse Fourier Transformation
\[
  f(x) = \int_{-\infty}^\infty F(u) e^{2 \pi i x u} \funv{d}u
\]

2D-Fourier-Transformation
\[
  FT_2(f)(u, v) = \int_{-\infty}^{\infty} \int_{-\infty}^{\infty} f(x,y)e^{-2 \pi i (xu+yv)}\funv{d}x\funv{d}y
\]

\section{Entropie}

\[
  H = - \sum_{i=0}^N p_i \cdot \funv{ln}(p_i)
\]

\end{document}
://drive.google.com/file/d/0B41UhTfm4XbpWkU1VmJRQXNwNmM/view?usp=sharing
